
\begin{abstract}
This thesis investigates the development and evaluation of a multimodal emotion recognition system to advance human-robot interaction. The study addresses the research problem of enabling robots to accurately and efficiently interpret human emotions in real-time, a critical capability for applications in healthcare, education, and customer service. The system integrates Facial Emotion Recognition (FER) and Sentiment Analysis to enhance emotional understanding.


For FER, multiple face detection algorithms, including Tiny YOLO, YOLO, dlib, and Haar Cascade, were tested, followed by emotion recognition using MobileNetV2, ResNet50, and VGG16 models trained via transfer learning. These models were rigorously evaluated for accuracy, efficiency, and real-time performance on robotic platforms. In the domain of Sentiment Analysis, IBM Watson was leveraged to analyse vocal inputs, focusing on sentiment classification accuracy and response times. The study highlights a critical trade-off between accuracy and inference speed, identifying optimal model combinations for real-world applications.


The results demonstrate the feasibility of deploying multimodal emotion recognition systems in robotics, with implications for improving human-robot interaction. The findings underline the importance of balancing technical performance metrics to ensure practical usability. Future directions include evaluating the system in dynamic, uncontrolled environments with human participants, integrating on-robot sentiment analysis systems, and exploring the fusion of both modalities to enhance emotional interpretation. These contributions aim to advance the field of socially intelligent robotics and foster more engaging and empathetic interactions with humans.\\

    \noindent{}
    \textit{Keywords}: Facial Emotion Recognition, Sentiment Analysis, Social Robot, Multimodal Emotion Recognition.
\end{abstract}

\vspace{5cm}
\begin{center}
\vspace{2cm}
\textbf{Author}\\ Joshua Bamforth \\
\vspace{2cm}
{\textbf{Supervisory team:} Prof. Alessandro Di Nuovo, Dr. Jing Wang}
\end{center}